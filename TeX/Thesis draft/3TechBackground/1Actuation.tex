Consider a general system with nonlinear time-varying dynamics:
\begin{equation}
	\ddot{q}(t) = f\left(\dot{q}(t), q(t), t, u(t)\right)
\end{equation}
where $q(t)$ is a vector of generalised coordinates, $\dot{q}(t)$ is the vector of velocities of those coordinates and $\ddot{q}(t)$ is the vector of accelerations, each of size $n$, and $u$ is the vector of control inputs of size $m$. \\

If, as is typically the case in mechanical systems, the acceleration of the generalised coordinates is linear in the control input, this can be expressed as: \\
\begin{equation}
	\ddot{q}(t) = f_1\left(\dot{q}(t), q(t), t\right) + f_2\left(\dot{q}(t), q(t), t\right)u(t)
\end{equation}

For this class of mechanical systems, we say that the system is considered to be \textit{fully actuated} if and only if rank($f_2$) $=$ dim($q$). If the system is not fully actuated, it is considered to be \textit{underactuated}. If $f_2$ has zero rank, or if $u$ is empty, then the system is considered to be \textit{unactuated}. Note that whether or not the system is fully actuated is possibly both time and state-dependent.  \\

This mathematical formulation may be interpreted as the following statement:

\textit{A system is fully actuated if and only if the accelerations of the generalised coordinates of the system are able to be arbitrarily set through the application of control.}

Note that this is an idealised statement; it ignores the necessity to accurately model $f_1$ and $f_2$, and physical limits such as torque limits in motors, which by very nature disallow controls to arbitrarily affect systems. \\

The underactuated systems to which the planning algorithms of this thesis apply, i.e. underactuated dynamic walkers, can be understood to be systems which have at least one less control input than the number of generalised coordinates. Thus even if $f_2$ has full rank, its rank will still be at most dim$(q)$ - 1. It should be noted that systems which have actuators on all coordinates can still act as underactuated systems under certain conditions, since $f_2$ is state-dependent.