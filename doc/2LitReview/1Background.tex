This work is focused on path planning for underactuated walking robots using virtual holonomic constraints as motion primitives. The objective is to allow for robust real-time approximately-optimal control of underactuated walkers on rough terrain. This is based upon previous work in several fields, including walking robot dynamics, motion planning, the application of virtual constraints in control and path planning and feedback control to enforce the desired trajectories.

Walking robots are a subset of mobile robots. Other common methods of locomotion include wheels, rails and tracks. Walking robots have gathered interest for their ability to traverse terrain which is impassable to these other classes of mobile robot. However, walking robots are typically significantly less efficient and difficult to plan and control in comparison to wheeled robots. In order for legged locomotion to be viable, it must become robust, efficient and rapid. Also, other than for the limited applications in which motion can be pre-planned, it must be computable in real-time.

Feedback control has been a very important field of research in generating robust and useful robots. It has proven exceptionally effective at ensuring robotic motions match planned motions, particularly in systems of sufficiently low dimensionality and complexity. For conventional wheeled and fixed robots, feedback control laws are normally well-understood and behaviour can be prescribed which will be reliably and robustly tracked. However, for legged robots, especially underactuated robots, the required feedback control is not obvious. There has been a large amount of recent work in analysing the stability of control schemes in periodic and non-periodic walking cycles.

Motion planning has been a focus of robotics research from the beginnings of autonomous mobile robots. Simple conventional approaches which are applicable to wheeled robots have been developed which allow for very efficient motion planning to achieve particular goals within set of constraints. While the principles of motion planning are similar in legged robots, these techniques are most often not applicable due to the highly nonlinear, nonsmooth, nonconvex and discontinuous nature of legged walking. As such, there has been much work in recent times to address some of the problems of \textit{kinodynamic} planning; that is, planning which includes constraints on the velocities as well as the positions, as opposed to positions alone.

In the last decade, virtual constraints have been verified in the literature as a method for reducing the complexity of the required computation in solving the dynamics equations for underactuated robotic walkers \cite{wetervelt2003hybrid, shiriaev2005constructive, shih2007asymptotically}. The use of virtual constraints has allowed for potentially high-dimensional dynamics to be reduced to dynamics in a single variable, which makes for significantly simpler analysis.

This literature review will explore the classes of walking robots and motivate the need for developing controllable and robust underactuated walkers. Feedback control schemes will be explored, with a particular emphasis on those applicable to underactuated walkers. Motion planning methods and architectures are discussed, with a focus on the recent innovations in using constraints-based planning. The recent advancements in the understanding of the utility and properties of virtual holonomic constraints are expounded, with a view to implementing them as motion primitives for an underactuated dynamic walker. The literature survey is concluded by highlighting the state of the art of motion planning of underactuated robots, particularly those subject to virtual constraints, and explores the contributions of this thesis work in filling needs for solutions in this field.