Motivated by the dexterity, terrain scalability and reach availed to humans through the use of legs, as opposed to wheels, and for social and aesthetic reasons, robotics researchers have sought to create humanoid robots. While there are notable examples of significant success in this endeavour, the motion characteristics of these robotic walkers are noticeably divergent from our own. This is largely due to the restricting of the robotic systems to full actuation, a domain which avails to the control system designer convenient simplifying properties and robustness. This restriction imposes more than simply visual differences; the conservative motion restricts the speed at which these robots can move and, importantly, the efficiency with which the walkers achieve motion. Due to these facts, there exists interest in developing underactuated robotic walkers, which utilise the dynamics of the system to generate much of the motion, imposing actuation only where required to enforce a trajectory or respond to disturbance. \\

However, the path planning and control problems become significantly more complex outside of the domain of fully actuated motion. This is mainly due to two contributing factors. Firstly, the dynamics of walking robots are highly nonlinear, particularly about the foot impacts, thus the system model must be very complex, or else be a poor approximation. Secondly, the search space to find feasible paths is vast, since in general robotic walkers have many independent degrees of freedom, making traditional methods of planning and control computationally intractable. In addition, control problems of underactuated and fully actuated control are sufficiently different that much of the intuition and methods from the quite well studied field of fully actuated control are not easily applicable to underactuated cases. \\

Manchester et al \cite{manchester13planning} have proposed a method for path planning for underactuated dynamic walkers using a receding-horizon algorithm which draws from a library of motion primitives, defined as virtual holonomic constraints, to address the above mentioned challenges. The use of a library of motion primitives drastically reduces the search space and notionally limits the trajectories to feasible motion, subject to the avoidance of collisions. The difficulty of the nonlinear impact dynamics is isolated by the choice of motion primitives as being defined by the duration of one continuous phase – that is, the movement of the robotic walker between one foot impact and the next. Also, the use of virtual holonomic constraints allows for a partial closed-form solution of the dynamical equations to be computed off-line, thereby reducing the required on-line computation. This method has been proven to be capable of planning several footsteps ahead of the current position of the robot in a fraction of a second, thus being feasible for real-time control, on a compass-gait walker and more complex 5-link walker. \\

The ability of this approach to produce paths close to the true optimal is dependent upon the library of motion primitives and the manner by which one primitive is preferred over another.
The current algorithm uses a greedy best-first search, with demonstrable improvements made in utilising an energy heuristic. The efficiency with which the algorithm obtains its result and its convergence to the most energy-efficient feasible path are the key performance indicators for the algorithm. Algorithmic efficiency is directly related to the required on-line computation. A less efficient algorithm will therefore consume more power (and time) to produce its result. Also, the power required to achieve the desired motion of the robot will clearly be greater if the path is less energy efficient. Naturally, more efficient algorithms which produce results which require less energy to achieve motion will reduce power consumption on board the robotic walker. Since power and energy densities pose strict limits on the dimensions and weight of walking robots with present technology, and due to ethical considerations in sustainable design, reducing energy consumption carries great importance. \\

As such, the work of this thesis is to refine the algorithms to enable efficient and robust real-time path planning of underactuated dynamic walkers. The algorithms demonstrably improve upon the current algorithms in running time and better convergence with optimal trajectories. This is realised by the design of data structures which are compatible with the implicit orderings of motion primitives paired with intelligent partitioning to enable an average case search time of $O(\log{n})$. The path planning algorithm is improved such that the traversal of the decision tree down infeasible paths is guaranteed to be minimised and the choice of motion primitives is guaranteed to produce more energy-efficient motion in comparison to the current methods. This is achieved in a two-fold manner. Firstly, the library of motion primitives is generated automatically to optimally span the configuration space within some bounds. Secondly, the algorithm to choose between primitives is improved with an intelligent heuristic that is informed by the shape of the upcoming terrain. The algorithms are verified by simulation on a compass-gait and more complex walkers, and by experiment on a compass-gait walker.