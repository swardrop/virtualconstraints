Recall from Section \ref{sec:zerodyn}, that the zero dynamics have the form
\[
\alpha\left(\theta\right)\ddot{\theta}(t) + \beta\left(\theta\right)\dot{\theta}(t)^2 + \gamma\left(\theta\right) = 0
\]
with
\begin{align*}
	\alpha(\theta) &= B^{\bot}\left(\Phi(\theta)\right)M\left(\Phi(\theta)\right)\Phi'(\theta)\\
	\beta(\theta) &= B^{\bot}\left(\Phi(\theta)\right)\left(M\left(\Phi(\theta)\right)\Phi''(\theta)
	+C\left(\Phi(\theta),\Phi'(\theta)\right)\Phi'(\theta) \right)\\
	\gamma(\theta) &= B^{\bot}\left(\Phi(\theta)\right)G\left(\Phi(\theta)\right)\\
\end{align*}
Using $\theta$ as the dependent variable in place of time produces the differential equation
\[
\frac{d}{d\theta}{\dot{\theta}\left(\theta\right)}^2 = -2\frac{\beta\left(\theta\right)}
{\alpha\left(\theta\right)}\dot{\theta}\left(\theta\right)^2 - 2\frac{\gamma\left(\theta\right)}
{\alpha\left(\theta\right)}
\]

Noting that his equation is a first-order ODE with varying coefficients, this may be solved by using the general method for differential equations of this class, i.e. given
\begin{equation*}
y'(x) + f(x)y(x) = g(x)
\end{equation*}
the solution is
\begin{equation*}
y = e^{-\int f(x)dx}\left(\int g(x)e^{\int f(x)dx}dx + \kappa\right)
\end{equation*}

Thus we may produce the partial closed form solution by solving the DE over the interval $\theta \in \left[\theta_0,\theta^-\right]$
\[
\dot{\theta}\left(\theta\right)^2 = \Gamma\left(\theta\right)\dot{\theta}^2 
+ \Psi\left(\theta\right)
\]
where
\begin{subequations}
	\begin{align}
	\Gamma(\theta) &= e^{-\int_{\theta_0}^{\theta}f(x)dx} \\
	\Psi(\theta) &= e^{-\int_{\theta_0}^{\theta}f(x)dx}
	\int_{\theta_0}^{\theta}g(x)e^{\int_{\theta_0}^{\theta}f(x)dx} \\
	f(x) &= 2\frac{\beta(x)}{\alpha(x)} \\
	g(x) &= -2\frac{\gamma(x)}{\alpha(x)}
	\end{align}
\end{subequations}

In order to produce the functions $\Gamma(\theta)$ and $\Psi(\theta)$ defining the coefficients of the partial solution, it is necessary to sample $\theta$ values on the interval $\left[\theta_0,\theta^-\right]$. This could be achieved in several ways. A brute-force method would be to calculate the integral from $\theta_0\to\theta$ for every $\theta$ in a grid. Since this method would use MATLAB's inbuilt libraries for integration, it would be accurate. However, this is a very inefficient means of solving the integrals since it involves recalculation of the same portion of the integral many times.

Therefore, the method of producing the partial solution used in this work is to sample the functions $f(\theta)$ and $g(\theta)$ over a sufficiently fine grid between $\theta_0$ and $\theta^-$ and to use cumulative trapezoidal integration to generate the value of $\Gamma(\theta)$ and $\Psi(\theta)$. Note that this is susceptible to sampling error but produces the result significantly more rapidly than individual calculations of the integral for many $\theta$ values.