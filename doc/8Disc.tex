\section{Key observations}
This will be somewhat dependent on the results, but I expect the most important points will be:
\begin{itemize}
	\item Velocity-independent orderings possible/not possible. If possible, the flexibility of the library is significantly improved
	\item Zero crossings of $\alpha(\theta)$, being the significant issue in the corruption of the optimisation, are/aren't found in useful constraints
	\item Virtual constraint generation is, apart from a single point of synchronisation, embarrassingly parallel and therefore scalable
	\item Constraints optimised for a particular initial velocity suffer badly/are still near-optimal for other velocities in the operational limits of the robot
	\item Importance of richness of kinetic energy mutations per configuration?
	\item Importance of richness of configurations per step length and height?
\end{itemize}

\section{Advantages of virtual constraints method}
\begin{itemize}
	\item Reduces infinite dimensional nonlinear optimisation to combinatorial search
	\item Directly corresponds to configuration path; allows simple collision checking
	\item Partial solutions
	\item Ordering
	\item Allows for on-board computer to be less powerful and to spend more CPU time on other things
	\item Allows for pre-vetting of motions by design of the library
\end{itemize}

\section{Limitations of virtual constraints method}
\begin{itemize}
	\item Lack of spontaneous planning
	\item Number of VCs grows with $\lvert\tilde{Q}\rvert^2$
	\item Limited in general optimality - can only optimise for single initial velocity
	\item Has (at present) no method of recovery
	\item Not stabilising in $\theta$ (pure VC)
\end{itemize}