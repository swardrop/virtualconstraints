This thesis work is directed at the generation of stable \textit{dynamic} walking for bipedal robots. This is a significantly more difficult problem than \textit{static} walking. Static walking is the application of legged locomotion such that at all times the robot is statically stable. Such robots were built as early as the 1980's, see e.g. \cite{russell1983odex, waldron1986adaptive}. It is worth noting that statically stable walking is only achievable in robots with at least four legs.

Interest in dynamic walking arises due to the fact that static walkers operate with restricted speed and efficiency in order that inertial effects remain minimal. Bipedal walking is especially of interest due to its similarity with human locomotion. There has been recent work in applying principles revealed in the study of bipedal dynamic walkers to improving rehabilitation of amputees with prosthetic limbs \cite{martinpredicting}. Bipedal walkers can be classed into three different broad types on the basis of how they are controlled; fully actuated, underactuated and passive.

\subsection{Fully actuated bipedal walking robots}
Full actuation presents a number of simplifying properties that make path planning and control design significantly easier than for underactuated systems. As a result, control of walking robots is made much simpler if the robots can remain fully actuated. This is the mode of control naturally applicable to static walking. It is possible to ensure bipedal robots remain in the realm of full actuation by restricting the motion such that the centre of pressure remains at all times underneath one of the robot's feet.

Possibly the most prominent example of a fully actuated bipedal walking robot is Honda's ASIMO robot. In \cite{chestnutt2005footstep}, the process by which footsteps are planned for ASIMO are detailed. While not a trivial process, the choice of footstep placement is driven primarily by kinematic concerns. There are dynamic constraints on the robot which exist in order that it remains within the realm of full actuation, but within those constraints, footstep placement can be arbitrary. \cite{chestnutt2007locomotion} outlines the robot's ability to extend this to predictably moving obstacles. In both examples footstep placement is achieved by employing the A* dynamic programming algorithm searching through a gridded map of the environment along with a set of possible actions that the robot's swing leg can achieve.

However, the ease with which fully actuated robots are controlled and guided come at significant cost; there are many sources, e.g. \cite{mcgeer1990passive, asano2007dynamic, byl2008approximate}, which suggest that energy-efficient walking is achievable in imitating passive dynamic walking as closely as possible. That is, there is significant consensus that in order to achieve energy-efficient gaits, underactuated control must be utilised. This is intuitively clear; restricting motion to full actuation is in many ways similar to restricting the robot to static stability. In fact, in many cases the two are equivalent. From trivial observation of human gaits, we understand that efficient, rapid locomotion is generated in gaits which significantly diverge from fully actuated control, especially during accelerated motion such as running or jogging, where a large proportion of the motion is without contact with the ground.

\subsection{Passive dynamic walkers}
Interest in studying passive dynamic walkers as a key to unlocking efficient bipedal walking and better understanding human gaits was sparked by McGeer's seminal paper on the subject \cite{mcgeer1990passive}. A class of mechanical systems for which there are natural stable periodic gaits which require no active control were introduced. This includes walking robots which have stiff legs as well as those which have articulated knees. McGeer validated the theory through experiment with a stiff-legged biped which was designed with four legs -- outer legs were connected by a crossbar and the inner legs were fixed together such that the motion is constrained to two dimensions, simulating the theoretical saggital-plane (side-on) compass gait walker.

In \cite{collins2001three}, a more sophisticated passive dynamic walking robot was developed which verified by experiment the generalisation of McGeer's work to three-dimensional kneed bipedal walkers. This robot also had arms which provided counterweights to improve dynamic stability. The robot was able to generate stable walking gaits on a $3.1^{\circ}$ slope, consuming 1.3 W (provided by the gravitational potential field) to achieve 0.51 m/s forward-motion. This may be compared to its contemporary Honda P3 robot, which requires 2 kW during walking.

However, passive dynamic walkers cannot feasibly provide reliable walking motions for general mobile robots, since they require downhill trajectories and are unable to be controlled, other than in setting initial conditions. They also suffer from high sensitivity to disturbances and require initialisation close to their dynamic equilibrium. For example, the Collins 3D walker was only able to be successfully launched 80\% of the time, even with a practised hand. It also suffered from low directional stability, unable to complete a 5m walk down a straight and narrow ramp without falling off in the majority of cases.

\subsection{Underactuated dynamic walkers}
The problems of a lack of controllability and robustness in passive dynamic walkers as well as a lack of efficiency in fully actuated walkers seem to find their natural solution in underactuated dynamic  walking robots. Indeed, in both \cite{mcgeer1990passive, collins2001three}, the authors envisioned that the robots should be actuated to provide sufficient energy to walk on flat ground or climb hills. In \cite{tedrake2004actuating}, this was achieved; a simple stiff-legged 3D dynamic walker was actuated to allow for periodic walking up shallow slopes. Also in this walker, by engineering the feet such that the curve of the foot is higher than the centre of mass, standing still was made a statically stable configuration, a property that the above-mentioned passive walkers lack.

A slightly more mechanically complicated walker based upon the Collins passive dynamic walker is presented in \cite{collins2005bipedal}. Here, however, there are far more actuators present; they are used sparingly, to keep the motion close to the natural passive gait. It includes hip actuation as normal, ankle actuation used for pre-impact toe-offs and some active control to ensure that the knees lock and unlock at convenient times. The controller is very simple, implemented in 68 lines of C++ code as a set of switches to synchronise the actuators. The robot was designed to generate very efficient stable periodic walking which is more robust than in the passive dynamic walker, and to be able to walk on level ground. While the active control did produce the ability to walk on level ground, its robustness was still poor.

Much of the literature in underactuated dynamic bipedal walker control is, similarly to \cite{tedrake2004actuating, collins2005bipedal}, focused on the application of torques to achieve stable periodic motion, see e.g. \cite{grizzle2001asymptotically, shiriaev2005constructive, sreenath2011compliant}. This regulates the behaviour of the robot to the stable periodic cycles inherent within the dynamics of the unactuated walker. This method allows for the most energy-efficient stable walking possible, given the particular dynamics of the mechanical system, under the assumption of locomotion over somewhat flat ground. However, this approach does not provide a means of feasible and efficient traversal over uneven terrain, since the periodic cycles generated may collide with step-ups in the environment as well as fail to adjust the motion intelligently to increases and decreases in gravitational potential energy.

Recently, attempts have been made to produce underactuated walking control methods to allow for more optimal walking on rough terrain. In \cite{byl2008approximate}, this was achieved by using a receding-horizon approach, using a value iteration algorithm \cite{sutton1998introduction} to find an approximately optimal step-to-step feedback policy chosen from a mesh of post-collision states. This used a combination of PD control to regulate the hip angle and an impulsive pre-collision toe-off which was verified by simulation to allow the robot to negotiate complex terrain successfully.

Underactuated robotic control is normally executed primarily at the hip, see e.g. \cite{tedrake2004actuating, byl2008approximate, manchester2011stable}, however the use of ankle-only actuation has been studied in \cite{franken2008analysis}. Here, it is shown that by choosing particular mass distributions and foot shapes, stability and robustness of convergence to the limit cycle can be improved. An important result was also to show that pre-impact push-off of the stance leg improves mechanical energy efficiency by 25\%. Of course, this requires the existence of feet on the ends of the robot's legs. In \cite{asano2007dynamic}, it is demonstrated that the use of curved feet and judicious application of hip torque is sufficient to simulate the effects of actuated ankles. \cite{martin2014design} confirms the efficiency benefit of curved feet over point feet.

Underactuated systems may classified on the basis of an important measure; the \textit{degree} of underactuation. This is the number of generalised coordinates which describe the state of the system which are unactuated. Many of the applications of underactuated control in the literature deal with underactuation degree one systems, typically with the ankle angle being the single unactuated coordinate, e.g. \cite{byl2008approximate, westervelt2003hybrid}. However, this typically applies only to saggital-plane models or robots specifically designed to exhibit two-dimensional motion characteristics. Even simple compass-gait-like walkers such as in \cite{tedrake2004actuating} introduce many unactuated degrees of freedom when expanded to three-dimensional walking.

It is important to note that while underactuatuation in robotic walkers often implies that there are states which are not actuated by mechanical design, underactuation in general is dependent upon the state. That is, many of the examples of dynamic walkers are underactuated by virtue of having joints which do not have any corresponding control, in particular models of walkers with point-terminated legs, see e.g. \cite{westervelt2003hybrid}. However, underactuation includes a much broader class of robots and is based upon the control strategy. This is trivially proven by considering a robot with direct control of each joint for which the control strategy involves one or more zero-torque inputs. More generally, a robot enters the domain of underactuated control whenever the control inputs cannot arbitrarily alter the state evolution. This is more precisely mathematically defined in Section~\ref{sec:underactuatedMaths}.

The underactuated robots which form the focus for this thesis work are saggital-plane underactuation degree one robots with no ankle actuation. This matches a large proportion of the literature and provides simplifying assumptions which allow for the development of techniques without imposing large difficulties in generalising to three-dimensional walkers.