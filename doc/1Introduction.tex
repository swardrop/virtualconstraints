Motivated by the dexterity, terrain scalability and reach availed to humans through the use of legs, as opposed to wheels, and for social and aesthetic reasons, robotics researchers have sought to create humanoid robots. While there are notable examples of significant success in this endeavour, the motion characteristics of these robotic walkers are noticeably divergent from our own. This is largely due to the restricting of the robotic systems to full actuation, a domain which avails to the control system designer convenient simplifying properties and robustness. This restriction imposes more than simply visual differences; the conservative motion restricts the speed at which these robots can move and, importantly, the efficiency with which the walkers achieve motion. Due to these facts, there exists interest in developing underactuated robotic walkers, which utilise the dynamics of the system to generate much of the motion, imposing actuation only where required to enforce a trajectory or respond to disturbance.

However, the path planning and control problems become significantly more complex outside of the domain of fully actuated motion. This is mainly due to two contributing factors. Firstly, the dynamics of walking robots are highly nonlinear, particularly about the foot impacts, thus the system model must be very complex, or else be a poor approximation. Secondly, the search space to find feasible paths is vast, since in general robotic walkers have many independent degrees of freedom, making traditional methods of planning and control computationally intractable. In addition, control problems of underactuated and fully actuated control are sufficiently different that much of the intuition and methods from the quite well studied field of fully actuated control are not easily applicable to underactuated cases.

Manchester et al \cite{manchester13planning} have proposed a method for path planning for underactuated dynamic walkers using a receding-horizon algorithm which draws from a library of motion primitives defined as virtual holonomic constraints to address the above mentioned challenges. The use of a library of motion primitives drastically reduces the search space and notionally limits the trajectories to feasible motion, subject to the avoidance of premature collisions with the ground. The difficulty of the nonlinear impact dynamics is isolated by the choice of motion primitives as being defined by the duration of one continuous phase – that is, the movement of the robotic walker between one foot impact and the next. Also, the use of virtual holonomic constraints allows for a partial closed-form solution of the dynamical equations to be computed off-line, thereby significantly reducing the required on-line computation in determining dynamic feasibility. This method has been proven to be capable of planning several footsteps ahead of the current position of the robot in a fraction of a second, thus being feasible for real-time control, on a compass-gait walker and more complex 5-link walker.

Power and energy densities pose strict limits on the dimensions and weight of walking robots with present technology. Additionally, ethical considerations in sustainable design guide us to, where possible, reduce wasteful energy consumption. There are two applicable means by which the energy consumption of an underactuated walking robot may be reduced in the context of motion planning; minimising the actuator energy output and reducing the computational requirements of the planning algorithm. The virtual constraints approach avails a convincing means of achieving the latter, however the ability of this approach to produce paths close to the true optimal in terms of energy reduction is dependent upon the library of motion primitives and the manner by which one primitive is preferred over another.

In the light of this, the primary aim of this thesis is to produce a method by which optimal constraints may be generated in a procedure which ensures coverage over the configuration space of the robot. This library is designed to enable rapid searches through the structure. In addition to the creation of this library, the existing algorithm is improved such that the choice of primitives is more favourable. The combination of these two contributions enable efficient and robust real-time path planning of underactuated dynamic walkers. These contributions are verified by simulation on a simple 2-link walker in order to demonstrate the principles as clearly as possible. A 5-link walker based upon the RABBIT testbed \cite{chevallereau2003rabbit} is used to prove the scalability and generalisability of the library generation and algorithm to more complex robot models.