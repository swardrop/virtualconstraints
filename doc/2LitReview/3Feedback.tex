Control systems may be classed broadly in two different types: \textit{Open-loop} or \textit{feedforward} control and \textit{closed-loop} or \textit{feedback} control. Open-loop control involves calculating the control inputs required for a desired outcome in the actuated system and simply feeding those values directly to the actuators. Closed-loop control compares the behaviour of the system to some reference behaviour and applies corrective control. While feedforward control is simpler conceptually, it requires a very accurate model of the system in order to achieve the desired results. In systems such as underactuated walkers which exhibit very complicated nonlinear dynamics, accurately modelling the system is both analytically and computationally difficult. Therefore, while some aspects of feedforward control can be used to improve the stability and responsiveness of controllers, feedback control is necessary to enforce the system's compliance with its target trajectory.

The most simple and universal form of feedback control is PID (Proportional Integral Differential) control \cite{aastrom2001future}. It is a very simple controller, applying simple linear gains on the error, its derivative and its integral, but is very powerful and can be applied to most control problems. PID controllers are very well studied; as early as 1986, there were very comprehensive design rules and guides on how to apply PID control to systems with many different models \cite{rivera1986internal}. %
\nomenclature[*]{PID}{Proportional Integral Differential -- a ubiquitous form of control involving gains on the three named error components}
\nomenclature[*]{PD}{Proportional Differential -- As with PID, but without an integral term}

However, in the case of underactuated dynamic walkers operating under virtual constraints (see Section~\ref{sec:virtconstraint}), PID control does not typically track the desired trajectory efficiently or reliably. In order to achieve satisfactory tracking, a more intelligent choice of controller is required. This may include some aspect of feedforward control in which the controller attempts to enforce the constraint based upon some knowledge of the system dynamics, with the feedback process merely correcting any deviation. This is broadly known as hybrid zero dynamics (HZD) control; the controller's internal dynamics are designed to match the zero dynamics of the system. This is shown to improve tracking in \cite{sreenath2011compliant} and is proven to be robust to reasonable model uncertainty in \cite{martin2014design} by applying a HZD controller with the assumption of point feet to a robot with curved feet. 

In \cite{manchester2011stable}, a provably stabilising controller for an underactuated robot traversing rough terrain based on a modified Model Predictive Control (MPC) method combined with the notion of \textit{transverse linearisation} is described and compared to a hybrid zero dynamics based controller. Importantly, the controller is designed for \textit{non-periodic} trajectories. This is significant, since most work in controllers for underactuated walkers, e.g. \cite{martin2014design, sreenath2011compliant, raibert2008bigdog} is primarily concerned with generating stable periodic gaits. Hybrid zero dynamics controllers are in general only provably stabilising for periodic gaits. This limits the ability of the robot to intelligently place its feet and to traverse rough terrain. The ability to reliably stabilise non-periodic gaits is important to the robustness and stability of a path planner such as the object of this thesis. %
\nomenclature[*]{MPC}{Model Predictive Control -- a control method in which the control input required to achieve a trajectory is calculated. Typically applies to a controller used repeatedly with a receding horizon}


This work is largely theoretical and provides improvements to path planning algorithms assuming that the constraints are perfectly regulated. It is understood that in practice, these constraints must be enforced by feedback control. Hybrid zero dynamics control will in theory perfectly regulate the configuration constraints requested by the motion planner, however, as this entirely ignores kinetic concerns, the assumptions made about impact conditions may be violated. Manchester et al's \cite{manchester2011stable} transverse linearisation controller alters the path such that the velocity constraints are well regulated at each impact. Given the existence of suitable alternatives to enforce the trajectories set by the planning algorithm, this thesis work will not place a large emphasis on control.