\subsection{State of the art}
For the past decade, virtual constraint based controllers have been used to generate orbitally exponentially stable motions due to the simplification of the dynamics afforded by the reduction of the potentially high-dimensional, nonsmooth, nonconvex and  nonlinear relations to a single nonlinear equation in the phase variable, along with the partial closed-form solution available in many cases. These controllers have been proven to be effective in domains of underactuated systems that previously were inaccessible due to the computational difficulty of analysing the system dynamics. 

A key focus area of these advancements has been in developing underactuated dynamic walking robots of various complexity, from simple compass-gait walkers to higher degree of freedom robots such as the Rabbit 7-DOF walker. These systems often pose a more significant challenge due to the necessity of modelling the hybrid zero dynamics, imposed by the impact conditions. There are many examples of hybrid zero dynamics based controllers generating robust and efficient walking gaits. 

Recent contributions by Mancester et al have been to introduce path planning to hybrid zero dynamics based walking robots, such that the robots may be able to adjust their behaviour ahead of time based upon the terrain. This extends the concepts of virtual constraints generating periodic motions to allow for each footstep to be characterised by a possibly unique constraint. The use of virtual constraints avails to motion planning the same benefits as control design; a partial closed-from solution of the zero dynamics and a means of ordering sets of constraints. This allows for very fast online selection of motion primitives which has been shown to facilitate locomotion over uneven terrain.

\subsection{Current gaps in research}
The technique introduced by Manchester et al in \cite{manchester13planning} is dependent upon the generation of a library of motion primitives along with an algorithm to select the ``best'' primitive based upon data of the terrain ahead. Therefore, the best use of this technique will be one which presents some kind of optimal set of motion primitives to be selected, and at each footstep, chooses the most favourable primitive based upon some metric. 

The current method employed to generate the library of motion primitives is largely manual, with no guarantee of closeness to an optimal set. This presents two problems; the creation of motion primitives for high-DOF systems is not feasible under this method, and the set of primitives may not present coverage over the configuration space sufficiently to provide a motion primitive for every possibility which bears sufficient closeness to the optimum path. 

The current algorithms suggested to select motion primitives are not based upon a true notion of optimality. A best-first search algorithm is suggested which will find a feasible path over the horizon in which it operates, if one exits, but it may do so in a manner which is not energy efficient and may choose footsteps which result in being unable to pass terrain which would have otherwise been passable. A simple energy heuristic involving looking at the highest point in a receding horizon is also suggested in \cite{manchester13planning}. This heuristic has proven to provide better outcomes in adding energy to the system pre-emptively when an increase in height is ahead, but offers no guarantee of near-optimality, particularly on more varied terrain. 

\subsection{Contribution of this thesis work}
This thesis work extends the path planning approach raised in \cite{manchester13planning} by implementing a method by which the virtual constraint library can be automatically generated. This is scalable to high dimensionality and provides a measure of optimality over its coverage of the configuration space. A heuristic search based upon the best-first search algorithm from the previous work is also proposed, which more intelligently chooses a primitive based upon the upcoming terrain. The efficacy of the extensions to Manchester et al's work is demonstrated by validation with a simulated compass-gait walker as well as a more complicated 5-link walker.