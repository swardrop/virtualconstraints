\section{Contribution of this work to state of the art}
Synthesize advantages and disadvantages -- trajectory of field and its promise. Maybe here, at the end of this section?

\section{Future work}
The primary contribution of thesis work is to build upon the method of using virtual constraints as motion primitives for a path planning algorithm proposed in \cite{manchester13planning} by providing a means of procedural generation of a virtual constraint library. This method is, in its current form, still limited in its practicability due to the multiple simplifying assumptions made. In principle, the method can be generalised without fundamentally changing its structure.

The first and most obvious extension of this work is to generalise the method to being able to plan for 3D robot models. In principle, this is a fairly simple step; the virtual constraints are constructed to synchronise the additional degrees of freedom introduced by the extra spacial dimension to the phase variable in the same manner. However, the construction of the robot model and simulator become significantly more complex and the formulation of the motion primitives is more difficult to formulate.



\begin{itemize}
	\item 3D
	\item Experimentation/physical realisation
	\item High-level planner - starting and stopping, achieving some desired gait characteristics etc.
	\item Fail-safes
	\item Feet
	\item Optimisation of constraint with better specified path of end of foot?
	\item Bounding/running
	\item More efficient optimisation (in C, not MATLAB?)
	\item Impulsive toe-offs to improve efficiency?
	\item Produce constraints based on real-world ground data?
	\item Transverse linearisation controller
	\item Genuine flexible IK calculation for building of $\tilde{Q}$.
\end{itemize}