\section{Key observations}
In attempting to build a generalised method for the construction of virtual constraint libraries for the purpose of facilitating motion planning, this thesis has investigated the properties of individual as well as sets of virtual constraints. This investigation has been focused on those features which enable or obstruct the construction of a useful library which is able to be quickly traversed by a selection algorithm.

\subsection{Optimisation feasibility}
The first observation, which encourages the use of optimisation for the generation of the library, is that $p$-norms of the torque are well-shaped in terms of the decision variables (the Bézier coefficients). There is a notable exception to this caused by the numerical interaction of $\alpha(\theta)$ crossing zero and the sampling method used for numerical integration. While it is possible to improve the behaviour by adaptive sampling methods, the solution is susceptible to bad behaviour if the zero crossings are too close to each other in terms of the sampling interval. These zero crossings of $\alpha(\theta)$ form the most significant challenge to the general feasibility of the optimisation problem. It is noted that the behaviour of $\alpha(\theta)$ in terms of zero crossings worsens considerably with increasing values of the Bézier coefficients. Regularisation was used in this thesis as a method of guiding the optimisation away from areas of poor $\alpha(\theta)$ behaviour and avoiding the infeasibility of the constrained minimisation caused by its existence. Introducing this regularisation appears to have no undesirable side-effects in the optimisation.

\subsection{Ordering methods}
Prior to this work, since the method of producing virtual constraints was manual, it was not certain whether it is possible to construct a useful library of motion primitives which are able to be ordered in a manner which is independent of the target velocity. This is an important issue to investigate, since primitive selection algorithms which seek a variable velocity or energy have been shown to significantly reduce the number of traversals through the decision tree \cite{manchester13planning}. The results presented in this thesis are not conclusive, however they suggest that it may be possible to produce useful libraries with this structure with some effort. In the majority of cases, relatively few points in the relevant set violate the conditions for a velocity-independent ordering. This indicates that a library with the required structure may be able to be produced at the cost of sacrificing some of the coverage.

\subsection{Optimality}
In its conception, the virtual constraints method of path planning is concerned with producing feasible rather than optimal paths. This is not unique; in the space of nonlinear high-dimensional constrained optimisations, methods are typically some form of searching for a ``sufficiently good'' feasible path which we may boldly call \textit{approximately optimal} \cite{kavraki1996probabilistic, hwang1992potential, quinlan1993elastic}. This should not discourage us from searching for optimality wherever it can be achieved, particularly given that one of the main motivations for investigating underactuated robots is the increase in efficiency (decrease in required input power) in comparison to the more stable and easier to control fully actuated bipeds. It is therefore still relevant to discuss the optimality of the virtual constraints produced by the methods introduced in this thesis. 

The most significant potential detractor from this optimality is related to the fact that the torque required to maintain a virtual constraint is dependent on the velocity. The results of this work expressed in Figure \ref{fig:singleflattorque} suggest that the torque curves required to maintain a constraint are similar nearby the nominal velocity, therefore it seems safe to assume that the optimality of the constraint is roughly maintained over a region about the nominal velocity. The optimality of a planned path is therefore dependent on the choice of nominal velocity and how closely the velocity is to nominal when the virtual constraint is chosen by the path planner.

This dependence couples with the more general reliance on the efficacy of the selection algorithm. In principle, this thesis is concerned with the generation of the primitives much more than their selection, however a primitive library is entirely limited in its usefulness by the planner which uses it. The power of the planning algorithm is also limited by the data that is presented by the virtual constraint generation. Therefore, while it is true to say that this method of library generation is applicable to many selection algorithms, the optimality under the suggested algorithms should be assessed. {\color{red} Results from simulation discussion.}

\subsection{Off-line computational requirements}
Prior to this work, the in-principle low on-line computational costs of using virtual constraints had been discussed \cite{manchester13planning}. The computational requirements for preparing the library had not. It should be noted that the on-line computational requirement is key for the success of the real-time performance of the algorithm. However, if the library takes an infeasible amount of time to produce, these on-line performance metrics count for little. Fortunately, there are several key properties of this method of library production which suggest that, even for extremely large libraries, the computation should be tractable. The first is that the generation of the virtual constraint library is able to be executed in parallel, besides the single point of synchronisation after producing the set of impact configurations $\tilde{Q}$. This means that the library generation can utilise the full power of multi-core and distributed computing very effectively. It is also notable that the time to generate a single virtual constraint is able to be kept reasonably small. Of course, for more complicated robots, this grows, however the results appear to suggest that the complexity of the optimisation problem is quadratic in the number of decision variables, which is significantly more favourable than the exponential growth observed in the size of the search space. Another very compelling notable advantage in terms of preparation time is that it is only necessary to have a fine resolution over one of the dimensions $n_x$ and $n_y$. In addition, it is likely not necessary to produce a large number final configurations per step length and height. This limits the growth in complexity with increasing degrees of freedom in the robot model.

\section{Advantages of virtual constraints method}
\begin{itemize}
	\item Reduces infinite dimensional nonlinear optimisation to combinatorial search
	\item Directly corresponds to configuration path; allows simple collision checking
	\item Partial solutions
	\item Ordering
	\item Allows for on-board computer to be less powerful and to spend more CPU time on other things
	\item Allows for pre-vetting of motions by design of the library
\end{itemize}

\section{Limitations of virtual constraints method}
\begin{itemize}
	\item Lack of spontaneous planning
	\item Number of VCs grows with $\lvert\tilde{Q}\rvert^2$
	\item Limited in general optimality - can only optimise for single initial velocity
	\item Has (at present) no method of recovery
	\item Not stabilising in $\theta$ (pure VC)
\end{itemize}