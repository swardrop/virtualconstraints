\section{Outcomes of study}
Maybe something clever can go here

\section{Advantages of virtual constraints method}
And here
\begin{itemize}
	\item Reduces infinite dimensional nonlinear optimisation to combinatorial search
	\item Partial solutions
	\item Ordering
	\item Allows for on-board computer to be less powerful and to spend more CPU time on other things
	\item Allows for pre-vetting of motions?
\end{itemize}

\section{Limitations of virtual constraints method}
Maybe not too much here
\begin{itemize}
	\item Lack of spontaneous planning?
	\item Number of VCs grows with $\lvert\tilde{Q}\rvert^2$
	\item Limited in general optimality - can only optimise for single initial velocity
	\item Has (at present) no method of recovery
	\item Not stabilising in $\theta$ (pure VC)
\end{itemize}

\section{Future work}
Go wild here.

\begin{itemize}
	\item 3D
	\item Experimentation/physical realisation
	\item High-level planner
	\item Fail-safes
	\item Feet
	\item Optimisation of constraint with better specified path of end of foot?
	\item Bounding/running
	\item More efficient optimisation (in C, not MATLAB?)
	\item Impulsive toe-offs to improve efficiency?
	\item Produce constraints based on real-world ground data?
	\item Transverse linearisation controller
\end{itemize}